\section{Ziele der Implementierung}
\label{sec:implementation}

Die Hauptaufgabe des Praktikums besteht in der Realisierung des in \refnice{Abschnitt}{sec:design}
beschriebenen Programms, sowie der Erstellung der relevanten Auswertungen.
Die bereitzustellenden Aspekte sind dabei in Pflicht- und Optionale Bereiche aufgegliedert,
die im folgenden beschrieben werden.

\subsection{Pflicht: Vergleich verschiedener Strategien}

Das im vorigen Abschnitt beschriebene Vorgehen beruht auf der Idee, dass verschiedene
Implementierungen einer Ähnlichkeitsfunktion zur Entity-Resolution austauschbar
in dem gleichen Workflow ausgeführt werden können.
Um dies zu gewährleisten, wird insbesondere auch der Import der Daten durch die Ähnlichkeitsfunktion so beeinflusst,
dass beim Import die Daten bereits in der von der verwendeten Strategie bevorzugten Datenstruktur bereitgestellt werden.

\refnice{Tabelle}{fig:implementation} beschreibt alle Implementierungen von Ähnlichkeitsfunktionen, die wir im Rahmen der Aufgabe vergleichen wollen.
Viele Ansätze sind redundant vertreten, da wir der Auffassung sind,
dass kleine Implementierungsdetails schwer abzuschätzende Auswirkungen haben können.
Des weiteren existieren verschiedene Implementierungen, die entweder keine oder unvollständige Berechnung ausführen.
Die Motivation dafür besteht in der Möglichkeit, den Workflow hinsichtlich eines isolierten Aspekts zu bewerten.

\begin{figure}[h!]
\begin{tabular}{|l|}
	\hline
	\begin{minipage}{0.999999\textwidth}
		\vspace*{0.1cm}
		\textbf{Komponentenbasierte Personen-Ähnlichkeit}\\

		Ähnlichkeit entspricht arithmetischem Mittel der Jaccard-Ähnlichkeiten der
		N-Gramm-Mengen der Geburtstags-, Name-, Address-Komponenten.\\

		\hspace*{-0.4cm}
		\begin{footnotesize}
		\begin{tabular}{l}
			\hline
				\textbf{\underline{konstante Abbildung auf $0$:}}\\
				\begin{minipage}{0.999999\textwidth} Fokus: Bewertung Import + Transformation \end{minipage}\\
			\hline
				\textbf{\underline{Jaccard-KISS-Ähnlichkeit:}}\\
				Fokus: Lesbarkeit\\
			\hline
				\textbf{\underline{Jaccard-Ähnlichkeit:}}\\
				Fokus: Performance\\
			\hline
				\textbf{\underline{Dritt-Bibliothek-Bloom-Ansatz:}}\\
				Vergleich: Eigenimplementierung vs Verbreitete\\
			\hline
				\textbf{\underline{Bloom-Ansatz (Eigenimplementierung):}}\\
				Mit BitSet\\ 
			\hline
				\textbf{\textbf{Optional: Bloom-Ansatz (Eigenimplementierung):}}\\
				Mit Long-Array\\ 
			\hline
				\textbf{\underline{Optional: Bloom-Ansatz (Eigenimplementierung):}}\\
				Mit Boolean-Array\\ 
		\end{tabular}
		\end{footnotesize}
	\end{minipage}\\
	\hline
	\hline
	\begin{minipage}{0.999999\textwidth}
		\vspace*{0.1cm}
		\textbf{Personen-Ähnlichkeit}\\

		Alle Komponenten einer Person werden in einer gemeinsamen Menge aufbewahrt, wobei die Jaccard-Ähnlichkeiten
		dieser Menge die Ähnlichkeit der Personen entspricht. Ähnlichkeitsfunktionen haben die Bedeutung von oben.\\

		\hspace*{-0.4cm}
		\begin{footnotesize}
		\begin{tabular}{l}
			\hline
				\begin{minipage}{0.999999\textwidth}konstante Abbildung auf $0$\end{minipage}\\
			\hline
				Jaccard-KISS-Ähnlichkeit\\
			\hline
				Jaccard-Ähnlichkeit\\
			\hline
				Dritt-Bibliothek-Bloom-Ansatz\\
			\hline
				Bloom-Ansatz (Eigenimplementierung)\\
			\hline
				Optional: Bloom-Ansatz (Eigenimplementierung)\\
			\hline
				Optional: Bloom-Ansatz (Eigenimplementierung)\\
		\end{tabular}
		\end{footnotesize}
	\end{minipage}\\
	\hline
\end{tabular}
\caption{Übersicht der zu implementierenden Ähnlichkeitsfunktionen}
\label{fig:implementation}
\end{figure}

\subsection{Optional: Parallelisierung}

\begin{itemize}
	\item Ausführen des Workflows in Flink
\end{itemize}
